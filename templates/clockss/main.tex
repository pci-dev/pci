%%%  TO FORMAT A PREPRINT To submit to a PCI.  If accepted then use this main.tex with the package of accepted article and search for comments  %%%

\documentclass[a4paper]{article}

%%%   SET THE TITLE   %%%
\newcommand{\preprinttitle}{[[ARTICLE_TITLE]]}
%%%%%%%%%%%%%%%%%%%%%%%%%

%%%  SET THE Recommendation text.%%%
\newcommand{\recomtext}{[[ARTICLE_ABSTRACT]]}
%%%%%%%%%%%%%%%%%%%%%%

%%%%%%%%%SET the references of the recommendation text%%%%%%%%%%%%%
\newcommand{\references}{
[[ARTICLE_DATA_DOI]]
}


%%% UNCOMMENT TO CHOOSE A PCI %%%

\newcommand{\PCI}{[[PCI_COMMAND]]}
%%%%%%%%%%%%%%%%%%%%%



%%%  SET THE RECOMMENDER(s) NAME(s)  %%%
\newcommand{\recommender}{[[RECOMMENDER_NAMES]]}
%%%%%%%%%%%%%%%%%


%%%  SET THE REVIEWERS' NAMES IF KNOWN and/or X anonymous reviewers  %%%
\newcommand{\reviewers}{[[REVIEWER_NAMES]]}
%%%%%%%%%%%%%%%


%%% SET THE 'CITE AS' of the recommendation   %%%
\newcommand{\citeas}{[[ARTICLE_AUTHORS]] ([[ARTICLE_YEAR]]) [[RECOMMENDATION_TITLE]]. [[PCI_NAME]], [[RECOMMENDATION_DOI_ID]], [[RECOMMENDATION_DOI]]}
%%%%%%%%%%%%%%%%%

%%% SET THE 'CITE AS' of the recommended preprint   %%%
\newcommand{\citeaspreprint}{[[RECOMMENDATION_CITATION]]}
%%%%%%%%%%%%%%%%%


%%%%%%%%%%TO FORMAT A PREPRINT%%%%%%%%%%%%%%%%
%%%%%%%%%%%%%%%%%%%%%%%%%%%%%%%%%%%%%%%%%%%%%%
%%%%%%%%%%%%%%%%%%%%%%%%%%%%%%%%%%%%%%%%%%%%%%

\usepackage[margin=1in]{geometry}

\usepackage{marginnote}
\usepackage{setspace}
\usepackage{ifthen}
\usepackage{pdflscape}
\reversemarginpar  % sets margin notes to the left
\usepackage{lipsum} % Required to insert dummy text
\usepackage{calc}
\usepackage{siunitx}
\usepackage{lineno}
\usepackage{titlesec}
\usepackage{indentfirst}
\usepackage{amsmath}
\usepackage{amssymb}
%\usepackage[none]{hyphenat} % use only if there is a problem
% Use Unicode characters
\usepackage[utf8]{inputenc}
\usepackage[T1]{fontenc}
% Clean citations with biblatex
%\usepackage[
%backend=biber,
%natbib=true,
%sortcites=true,
%defernumbers=true,
%style=authoryear,
%citestyle=authoryear-comp,
%maxnames=99,
%maxcitenames=2,
%uniquename=init,
%giveninits=true,
%terseinits=true,
%url=false
%]{biblatex}
%\DeclareNameAlias{default}{family-given}
%\renewcommand*{\revsdnamepunct}{} % no comma between family and given names
%\renewbibmacro{in:}{%
%  \ifentrytype{article}{}{\printtext{\bibstring{in}\intitlepunct}}} % remove 'In:' before journal name
%\DeclareFieldFormat[article]{pages}{#1} % remove pp.
%\AtEveryBibitem{\ifentrytype{article}{\clearfield{number}}{}} % don't print issue numbers
%\DeclareFieldFormat[article, inbook, incollection, inproceedings, misc, thesis, unpublished]{title}{#1} % title without quotes
%\usepackage{csquotes}
%\RequirePackage[english]{babel} % must be called after biblatex
%\addbibresource{sample.bib}
%\DeclareBibliographyCategory{ignore}
%\addtocategory{ignore}{recommendation} % adding recommendation to 'ignore' category so that it does not appear in the References
% Clickable references. Use \url{www.example.com} or \href{www.example.com}{description} to add a clicky url
%\usepackage{nameref}
%\urlstyle{same}


%\DeclareFieldFormat{doi}{\url{https://doi.org/#1}}

% Include figures
\usepackage{graphbox}  % loads graphicx ppackage with extended options for vertical alignment of figures
% Line numbers
%\usepackage[right]{lineno}
% Improve typesetting in LaTex
\usepackage{microtype}
\DisableLigatures[f]{encoding = *, family = * }
% Text layout and font (Open Sans)
\setlength{\parindent}{0.4cm}
\linespread{1.2}
\RequirePackage[default,scale=0.90]{opensans}
% Defining document colors
\usepackage{xcolor}
\definecolor{darkgray}{HTML}{808080}
\definecolor{mediumgray}{HTML}{6D6E70}
\definecolor{ligthgray}{HTML}{d9d9d9}
\definecolor{pciblue}{HTML}{74adca}
\definecolor{opengreen}{HTML}{77933c}
% Use adjustwidth environment to exceed text width
\usepackage{changepage}
% Adjust caption style
\usepackage[aboveskip=1pt,labelfont=bf,labelsep=period,singlelinecheck=off]{caption}
\usepackage[pdfborder={0 0 0},   
    colorlinks=true,
    linkcolor=blue,
    urlcolor=blue,
    citecolor=blue]{hyperref}  % sets link border to white


% Headers and footers

% Headers and footers
\usepackage{fancyhdr}  % custom headers/footers
\pagestyle{fancy}  % enables customization of headers/footers
\fancyhfoffset[L]{0.5cm}  % offsets header and footer to the left to include margin
%\ifnum\thepage=1 \setlength{\headheight}{120pt}\else\fi
\renewcommand{\headrulewidth}{0pt}
\renewcommand{\footrulewidth}{0pt}
% full logo on first page, then no logo on subsequent pages 
\lhead{\ifnum\thepage=1 \includegraphics[width=13.5cm]{\logoname}\else \fi}  
\chead{}
\rhead{}
\cfoot{\thepage}
\rfoot{}
% End Headers and footers

% DOI's
\newcommand{\DOIlink}{\href{https://doi.org/\DOI}{https://doi.org/\DOI}}
\newcommand{\DOIrecommendationlink}{\href{https://doi.org/\DOIrecommendation}{https://doi.org/\DOIrecommendation}}



%which logo to display
\ifthenelse{\equal{\PCI}{Peer Community In Registered Reports}}{\newcommand{\logoname}{logo_PDF_rr.jpg}}{}
\ifthenelse{\equal{\PCI}{Peer Community In Zoology}}{\newcommand{\logoname}{logo_PDF_zool.jpg}}{}
\ifthenelse{\equal{\PCI}{Peer Community In Ecology}}{\newcommand{\logoname}{logo_PDF_ecology.jpg}}{}
\ifthenelse{\equal{\PCI}{Peer Community In Animal Science}}{\newcommand{\logoname}{logo_PDF_animsci.jpg}}{}
\ifthenelse{\equal{\PCI}{Peer Community In Archaeology}}{\newcommand{\logoname}{logo_PDF_archaeo.jpg}}{}
\ifthenelse{\equal{\PCI}{Peer Community In Forest and Wood Sciences}}{\newcommand{\logoname}{logo_PDF_fws.jpg}}{}
\ifthenelse{\equal{\PCI}{Peer Community In Genomics}}{\newcommand{\logoname}{logo_PDF_genomics.jpg}}{}
\ifthenelse{\equal{\PCI}{Peer Community In Mathematical and Computational Biology}}{\newcommand{\logoname}{logo_PDF_mcb.jpg}}{}
\ifthenelse{\equal{\PCI}{Peer Community In Network Science}}{\newcommand{\logoname}{logo_PDF_networksci.jpg}}{}
\ifthenelse{\equal{\PCI}{Peer Community In Paleontology}}{\newcommand{\logoname}{logo_PDF_paleo.jpg}}{}
\ifthenelse{\equal{\PCI}{Peer Community In Neuroscience}}{\newcommand{\logoname}{logo_PDF_neuro.jpg}}{}
\ifthenelse{\equal{\PCI}{Peer Community In Ecotoxicology and Environmental Chemistry}}{\newcommand{\logoname}{logo_PDF_ecotoxenvchem.jpg}}{}
\ifthenelse{\equal{\PCI}{Peer Community In Infections}}{\newcommand{\logoname}{logo_PDF_infections.jpg}}{}
\ifthenelse{\equal{\PCI}{Peer Community In Evolutionary Biology}}{\newcommand{\logoname}{logo_PDF_evolbiol.jpg}}{\newcommand{\logoname}{logo_PDF_evolbiol.jpg}} % Default logo for test platforms %

\newcommand{\beginingpreprint}{

\vspace*{3cm}
\begin{flushleft}
\baselineskip=10pt

{\Huge
\fontseries{sb}\selectfont{\fontsize{12pt}{14pt}\preprinttitle}}
\end{flushleft}
\vspace*{0.25cm}
\begin{flushleft}
\Large
A recommendation by {\color{pciblue}{\recommender}} based on the peer reviews by {\reviewers} of the article:
\bigbreak
\fcolorbox{lightgray}{lightgray}{
\parbox{\textwidth - 2\fboxsep}{
\centering\large{\selectfont{\citeaspreprint}}\\
}}
\vspace*{0.5cm}

\small{\textbf{Cite the recommendation as:}}\\
\small{\citeas}
\vspace*{0.5cm}

\textbf{\large{\textsc{Recommendation text}}}
\end{flushleft}

\recomtext\\

\textbf{\emph{References: }}
\begin{flushleft}
\references
\end{flushleft}

}




\begin{document}
\beginingpreprint

\section*{\centering Editorial process}

[[RECOMMENDATION_PROCESS]]

% \subsection*{Round 2}
% \subsubsection*{Authors' response}
% This is an example of text. This realization is especially important because it can flip around our expectations about which species expand fast, and how to manage them. We tend to think of initial colonization and long-term abundance as two independent axes of variation among species or indeed as two ends of a spectrum, in the classic competition-colonization tradeoff. When both play into invasion speed, good dispersers might not outrun good competitors. This is useful knowledge, whether we want to contain an invasion or secure a reintroduction. \\ 

% \subsubsection*{Decision by {\recommender}}
% This is an example of text. This realization is especially important because it can flip around our expectations about which species expand fast, and how to manage them. We tend to think of initial colonization and long-term abundance as two independent axes of variation among species or indeed as two ends of a spectrum, in the classic competition-colonization tradeoff. When both play into invasion speed, good dispersers might not outrun good competitors. This is useful knowledge, whether we want to contain an invasion or secure a reintroduction. \\ 

% \subsubsection*{Reviews}
% \subsubsection*{Review by Enrico Fermi}
% In their study "When higher carrying capacities lead to faster propagation", combine mathematical analysis, Individual-Based simulations and experiments to show that various mechanisms can cause pushed fronts, whose speed increases with the carrying capacity K of the species. Rather than focus on one particular angle, the authors endeavor to demonstrate that this qualitative effect appears again and again in a variety of settings. \\

% \subsubsection*{Review by anonymous reviewer}
% It is perhaps surprising that this notable and general connection between K and invasion speed has managed to garner so little fame in ecology. A large fraction of the literature employs the venerable Fisher-KPP reaction-diffusion model, which combines local logistic growth with linear diffusion in space. This model has prompted both considerable mathematical developments and many applications to modelling real invasions. But it only allows pulled fronts, driven by the small populations at the edge of a species range, with a speed that depends only on their initial growth rate r.\\

% \subsection*{Round 1}
% \subsubsection*{Authors' response}
% This is an example of text. This realization is especially important because it can flip around our expectations about which species expand fast, and how to manage them. We tend to think of initial colonization and long-term abundance as two independent axes of variation among species or indeed as two ends of a spectrum, in the classic competition-colonization tradeoff. When both play into invasion speed, good dispersers might not outrun good competitors. This is useful knowledge, whether we want to contain an invasion or secure a reintroduction. \\ 

% \subsubsection*{Decision by {\recommender}}
% This is an example of text. This realization is especially important because it can flip around our expectations about which species expand fast, and how to manage them. We tend to think of initial colonization and long-term abundance as two independent axes of variation among species or indeed as two ends of a spectrum, in the classic competition-colonization tradeoff . When both play into invasion speed, good dispersers might not outrun good competitors. This is useful knowledge, whether we want to contain an invasion or secure a reintroduction. \\ 

% \subsubsection*{Reviews}
% \subsubsection*{Review by Enrico Fermi}
% In their study "When higher carrying capacities lead to faster propagation", combine mathematical analysis, Individual-Based simulations and experiments to show that various mechanisms can cause pushed fronts, whose speed increases with the carrying capacity K of the species. Rather than focus on one particular angle, the authors endeavor to demonstrate that this qualitative effect appears again and again in a variety of settings. \\

% \subsubsection*{Review by anonymous reviewer}
% It is perhaps surprising that this notable and general connection between K and invasion speed has managed to garner so little fame in ecology. A large fraction of the literature employs the venerable Fisher-KPP reaction-diffusion model, which combines local logistic growth with linear diffusion in space. This model has prompted both considerable mathematical developments and many applications to modelling real invasions. But it only allows pulled fronts, driven by the small populations at the edge of a species range, with a speed that depends only on their initial growth rate r.\\

\end{document}
